
In this thesis we presented the Isolated Device Driver (IDDR) system. The IDDR system is an operating system which provides isolation between a device driver and the Linux kernel components by running the device driver in the driver domain. The IDDR system is a re-implementation of Xen's isolated driver domain.

Isolated driver domains use Xen's split device driver mechanism. Originally split device drivers exploit an interrupt based approach to interdomain communication. We replaced the interrupt based approach with a spinning based approach. This spinning based approach has the potential to reduce interdomain communication overhead including the number of context switch. IDDR system trades CPU cycles for better performance.

The IDDR system will be advantageous if used with I/O intensive applications. I/O intensive system can afford to utilize the idle CPU cycles for the benefit of the performance. 

\section{Future Work}
\paragraph{Adaptive spinning : }
In current implementation the read request thread and read response thread spins for a constant amount of time. It will be interesting to see the impact on performance if adaptive spinning is used.