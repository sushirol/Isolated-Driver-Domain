
In this thesis we presented the Isolated Device Driver (IDDR) system,
which we designed and implemented. 
The IDDR system is an extension to an operating system which provides isolation between
a device driver and kernel components by running the device
driver in an isolated driver domain. The IDDR system is an independent re-implementation 
of ideas originally proposed for Xen's isolated driver domains.

Isolated driver domains use Xen's split device driver
mechanism, which exploits an interrupt-based approach to interdomain communication. 
In IDDR, we replaced the interrupt-based
approach with a spinning-based approach. This spinning-based approach
has the potential to reduce interdomain communication overhead, including
the number of context switches.   Our evaluation found that we were able
to achieve better throughput for a variety of device drivers and workloads.

We also evaluated the CPU utilization since spinning trades CPU cycles 
for better performance.  Spinning may represent a reasonable choice, 
particularly in environments where I/O performance is paramount and CPU 
capacity is plentiful.

