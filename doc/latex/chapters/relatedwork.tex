% VirtuOS~\cite{Nikolaev:2013:VOS:2517349.2522719} is a library level solution that allows processes to directly communicate with domain. 

This chapter briefly discusses about the work closely related to the IDDR System. Since our work is divided into two parts: 1. Implementation of the Driver Domain to improve the robustness of the system 2. Performance improvement of the inter-domain communication, this chapter is divided into two section. Section~\ref{robustness} discusses the work which improves the robustness of the system and Section~\ref{interdomain} discusses the different work which concentrates on improving the inter-domain communication.
\\[3mm]

\section{Reliability of the System}

The goal of the baseline implementation of the IDDR system is to increase the robustness of the operating system. In this section we describe research in the operating systems area which focuses on increasing the robustness of the system.

\subsection{Driver Protection Approches}

A faulty device driver causes significant number of failures in the Linux kernel~\cite{tanenbaum2006can, coverity}. In the past research in area of user-mode driver frameworks, hardware-based driver isolation and language-based driver isolation has contributed to isolate device drivers.
\\[3mm]
There have been several attempts to run device driver code from in the user-mode. These frameworks allow user-level programming and good fault isolation. However, they suffer from poor performance~\cite{armand1991give}. They also require re-writing of the existing device drivers, and lacks compatibility~\cite{Leslie+:jcst2005}. In order to achieve the good performance and compatibility, Microdrivers~\cite{Ganapathy:2008:DIM:1346281.1346303} split a device driver into two parts. In Microdrivers, Performance critical operations of the device driver runs in the kernel and the rest of the driver code runs in a user-mode process. 
\\[3mm]
Nooks~\cite{swift2005improving} maintains the monolithic kernel structure, and focuses on making device drivers less vulnerable. Nooks creates a lightweight protection domain around each device driver. The domain is created by wrapping a layer of protective software around a device driver. 
The wrapper monitors all interactions between the driver and the kernel. It protects the kernel from a faulty device driver. Nooks requires device drivers to be modified as their interaction with the kernel changes. 
\\[3mm]
SUD~\cite{Boyd-Wickizer+:atc2010} uses IOMMU hardware to safely isolate malicious device drivers by running them in Linux processes. It runs unmodified Linux device drivers, by emulating a Linux kernel environment in user-space. A faulty device driver hence cannot compromise the rest of the system.

\subsubsection*{Virtualization Based Approches}

Xen isolated driver domain~\cite{Fraser04safehardware} is a device driver isolation architecture presented by Xen. The isolated driver driver allows unmodied device drivers to be run in a separate domain and shared across operating system instances. Isolated driver domain protects individual OSs and the system, from driver failure. The base IDDR system is the re-implementation of the Xen isolated driver domain. 
\\[3mm]
LeVesseur et. al.~\cite{LeVasseur04UnmodifiedDriverReuse} presents a system to reuse unmodified device drivers and improve system realibility using virtual machines. A unmodified device driver is run with a kernel, in a separate virtual machine. The main goal of this system is to reuse the device driver across different operating systems. However, the system isolates faults caused by a device driver by running a device driver in separate virtual machine.

\subsection*{Other Approches}
\paragraph{kernel based:}
Some systems provide a better fault containment by disintegrating the kernel functionality as system components. Microkernels are an example of this approach. Microkernels provide fault containment by isolating the system components. In microkernsls Mach~\cite{Accetta+:usenix86} and L4~\cite{Liedtke+:sosp95}, only essential functionalities like memory management, interprocess communication, scheduling and low level device driversare implemented in the kernel and remaining system components like file system, process management are implemented as a user process.
\\[3mm]
Some of the recent work also shows the similarity in the abstraction provided by the hypervisor and microkernel. For example, Microvisor~\cite{Heiser+:acm10} is a kernel that satisfies the combined objectives of microkernels and hypervisors. Microvisor provides an abstraction of a virtual machine, where a guest OS schedule activities on one or more VCPUs.
\\[3mm]
Dune is a system that provides an application a direct and safe access to the hardware features such as page tables, tagged TLBs and ring protection. It uses virtualization hardware to isolate applications from each other. Dune delivers hardware interrupts directly to applications to improve the signal delivery performance. However, Dune does not isolate device drivers from each other.

\section{Inter-domain Communication}
In the past numerous work on inter domain communication mechanisms was presented. Xen split drivers is one of the inter domain communication approach of Xen hypervisor~\cite{Fraser04safehardware}. The xen split drivers has overhead becuase of numerous context switches in form of event channel interrupts. It also incurs an overhead due to data copy, page flipping~\cite{Zhang:2007:XHI:1516124.1516138}. Xen hypervisor also provides a UNIX domain socket like interface for high throughput interdomain communication on the same system called XenSocket~\cite{Zhang:2007:XHI:1516124.1516138}. XenSocket replaces the page flipping design of the split driver. However, XenSocket needs an existing socket interface APIs to be changed. 
\\[3mm]
Fido~\cite{Burtsev:2009:FFI:1855807.1855832} is a shared memory based inter domain communication mechanism. Fido implements the fast interdomain communication mechanism by reducing data copies in Xen hypervisor. In contrast our system improves the inter domain communication mechanism of Split device drivers by avoiding the context switches. Also, even though Fido removes overhead with zero copy, users have to sacrifice the security and protections guarantees for that.
\\[3mm]

\ifbool{toShowBibliography}{\bibliography{references}}{}