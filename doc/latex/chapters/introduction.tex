% TODO edit

A system is judged by the quality of the services it offers and its ability to function reliably. Even though reliability of operating systems has been studied for several decades, it remains a major concern today. The characteristics of operating systems which make them unstable are size and complexity. 
\\
Software reliability study shows that 6 to 16 number of errors/1000 executable lines can be found within a module\cite{Basili:1984:SEC:69605.2085}\cite{Tanenbaum06canwe}. Linux kernel has over 15 million lines of code. If we assume minimum estimate, Linux kernel has around 90,000 bugs. Researchers have shown that the error rate for device drivers is higher than the error rate for the rest of the kernel\cite{Chou:2001:ESO:502034.502042}. Considering the fact that the operating system predominantly consists of device drivers, the faults in device drivers make the operating system unreliable\cite{Chou:2001:ESO:502034.502042}.

\pagebreak

\section {Problem Statement}

In order to make a system reliable, it is essential that a device driver code does not contain any bug. However, finding all these bugs and fixing them is difficult since bug fixes introduces new lines of code resulting in new bugs. 
\\
In modern operating systems memory protection is a way to control memory access rights. The memory protection prevents a process from accessing memory that has not been allocated to it. The memory protection prevents a bug within a process from affecting other processes, or the operating system\cite{Denning:1970:VM:356571.356573}\cite{Galvin}. Monolithic kernel component do not have the same level of isolation the user level applications have. Unlike user applications, monolithic kernel has hundreds of procedures linked together. As a result, any portion of the kernel can access and potentially overwrite any kernel data strcuture used by an unrelated component. Such a non existant isolation between kernel and device driver causes a bug in device drivers to corrupt the memory of the other kernel components. This memory corruption might lead to system crash. Hence the underlying cause of unreliability in the operating system is the tight coupling between device driver and linux kernel.

\pagebreak
  
\section {Proposed Solution} 

The reliability of a system can be improved by executing device drivers in an isolated environment from the kernel. The dependency of device drivers on other kernel components such as memory management, scheduler etc. make it difficult to isolate device drivers from the kernel.
\\
Our solution here adapts the concept of 'virtulization based fault protection in operating systems' proposed by LeVasseur et. al. \cite{LeVasseur04UnmodifiedDriverReuse}, Tanenbaum et. al.\cite{Tanenbaum06canwe}, Nooks\cite{Swift:2002:NAR:1133373.1133393}, Soltesz et. al. \cite{Soltesz:2007:COS:1272998.1273025}. The solution we implement runs a special program called hypervisor. Hypervisor is commonly used to run multiple operating systems in parallel, by exploiting the hardware. The use of virtual machines has a well-deserved reputation for extremely good fault isolation. Since none of the virtual machines are aware of the other virtual machines, malfunctioning of one virtual machine cannot spread to the others.
\\
In a virtualized environment, all virtual machines run as separate user processes in different address spaces. Thus, to exploit the memory protection capability between virtual machines, we run a device driver with minimilastic kernel in one virtual machine, and run user applications and a kernel together in the other virtual machine. As a result, a device driver is isolated from the linux kernel, making it impossible for the device driver to corrupt any kernel data structure in the virtual machine running user applications. 

\pagebreak

\section{Core Contributions}

The core contributions of this project can be divided into two parts. 

\begin{enumerate}
\item The implementation of the device driver isolation concept using two approaches. 
\item The performance comparison of the approaches.
\end{enumerate}

The first approach implements device driver isolation using interrupt based communication between front end and back end device drivers. Xen uses the same approach for implementation of the driver domain\cite{driverdomain}. This interrupt based model requires context switches\cite{Barham:2003:XAV:945445.945462}. The second approach implements the communication between front end and back end device drivers using spinning of threads. The spinning based model does not require the context switches. This report presents the performance comparison of both the approaches. In addition, the report aims to find if the performance degradation can be attributed to the context switches.

\pagebreak
\section {Organization}

This section gives the organization and roadmap of the thesis.

\begin{enumerate}
\item Chapter 2 gives the background on memory protection, virtualization, Xen Hypervisor, inter-domain communication, processes and threads. 
\item Chapter 3 gives the introduction to design of the system to isolate device driver. 
\item Chapter 4 discusses the detailed design and implementation to isolate device driver. 
\item Chapter 5 evaluates the performance of Independent device driver with different designs.
\item Chapter 6 reviews the related work in the area of kernel fault tolerance.
\item Chapter 7 concludes the report and lists down the topics where this work can be extended.
\end{enumerate}

% ref
\ifbool{toShowBibliography}{\bibliography{references}}{}
