We use Linux Kernel version 3.5.0 for both application domain and driver
domains. We use the Arch Linux distribution for the \texttt{x86\_64}
platform for our testing and performance evaluation. The specifications of
the system used for the evaluation are presented in Table~\ref{tab:config}.

\begin{table}
\caption{Hardware specifications}
\begin{center}
\begin{tabular}{ll}
  \hline
  \label{tab:config}
  System Parameter & Configuration \\
  \hline
  Processor & 2 X Quad-core AMD Opteron(tm) Processor 2380, 2.49 Ghz \\
  Number of cores & 4 per processor \\
  Hyperthreading & OFF \\
  L1 L2 cache & 64K/512K per core \\
  L3 cache & 6144K \\
  Main memory & 16Gb \\
  Storage & SATA, HDD 7200RPM \\
  \hline 
\end{tabular}
\end{center}
\end{table}

\section{Goals}
\label{sec:goals}
The goals of our evaluation are:
\begin{enumerate}
\item \textbf{Comparison  of Xen's isolated driver domain mechanism with the interrupt-based IDDR system:}

In order to verify that the interrupt-based IDDR system constitutes a suitable baseline against 
which to compare the spinning-based IDDR system, we compared our interrupt-based implementation
against Xen's isolated driver domain.  However, since the source code
of the isolated driver domain implementation is not available, we achieve this evaluation goal 
by comparing the performance of the interrupt-based IDDR system with Xen's regular
drivers, which use the same split device driver model.

\item \textbf{Performance impact of spinning-based optimizations:}

The second goal of the evaluation is to prove that the spinning-based implementation of the 
communication channel improves the performance of the interdomain communication and hence the IDDR system.

We achieve this evaluation goal by comparing the performance of the interrupt-based IDDR system 
with the spinning-based IDDR system.
\end{enumerate}

\section{Methodology}
\label{sec:methodology}
In order to measure the performance of the system, we run performance
tests using a variety of block devices. In order to cover a variety of
devices we use block devices such as SATA disks, ramdisks and loop
devices for the performance testing. A loop device is a device that provides 
a block device that is backed by a file.  A ramdisk is a block of a memory
that acts as a disk drive.

In order to conduct the performance tests, we format the block
device with the ext2 file system, and run the SysBench
benchmark~\cite{sysbench} on it. SysBench is a multi-threaded benchmark
tool for evaluating operating system performance. It includes different test
modes, one of which is FileIO.  The FileIO mode can be used to produce
different file I/O workloads. It runs a specified number of threads to
execute requests in parallel. We run the SysBench benchmark in FileIO
test mode to generate 128 files with \texttt{1GB} of total data. We
execute random reads, random writes, and a mix of random reads and writes
on all three devices. We set the block size to \texttt{16KB}. We vary
the number of SysBench threads from 1 to 32 to measure the throughput of
the system under different levels of concurrent access.

\section{Xen Split Device Driver vs Interrupt-based IDDR System}
As per our first evaluation goal discussed in Section~\ref{sec:goals},
we compare the interrupt-based IDDR system with Xen's split device driver.

\subsection*{Experimental Setup}

\subsubsection*{Xen Split Device Driver}
We create a ramdisk in domain 0. The guest domain (domain U) is configured
to access domain 0's ramdisk via a split device driver. 
We use the same setup for the loop device and the SATA disk. 
We format and mount the disk into a partition in the
guest domain using the ext2 file system in all cases. The SysBench 
benchmark is run on the mounted partitions as explained in 
Section~\ref{sec:methodology}.

\subsubsection*{Interrupt-based IDDR System}
In the Xen hypervisor, the domain 0 always runs as a paravirtualized guest (PV),
but domain U can be using both a hardware-assisted virtualization configuration (HVM) 
or a paravirtualized configuration.  In our setup we run domain U always
in HVM mode because HVM guests exhibit less system call overhead
and faster memory bandwidth compared to PV guests, as we observed
by running the system call micro-benchmark tool lmbench\cite{lmbench}. 

When comparing to Xen's split device drivers, the backend device driver executes
in domain 0 and the frontend device driver in domain U.  
To eliminate any performance differences that may be solely due the mode
of virtualization used, we matched the modes of virtualization
by running IDDR's backend and frontend drivers also in 
domain 0 and U, respectively.

\subsubsection*{Comparison}
We compare the throughput of Xen's split device driver and the
interrupt-based IDDR system in Figure~\ref{fig:iddrvsxen-ramdisk-rdwr}
and Figure~\ref{fig:iddrvsxen-loop-rd}. 
Figure~\ref{fig:iddrvsxen-ramdisk-rdwr} presents the
throughput of both systems when data is randomly read from a ramdisk and
written to it. Figure~\ref{fig:iddrvsxen-loop-rd} presents the throughput
when data is randomly read from a loop device.

On a loop device, the performance of the interrupt-based IDDR system
differs by 3\%-4\% when compared to Xen's split device driver. On a
ramdisk, the throughput of the interrupt-based IDDR system matches that
of the Xen split device driver. This shows that our interrupt-based
IDDR implementation provides a suitable baseline for our performance 
optimizations.

\begin{figure}[!ht]
\centering
  \begin{subfigure}[b]{\textwidth}
  \includegraphics[width=\textwidth]{iddrvsxen-ramdisk-rdwr}
  \caption{Random reads and writes on a ramdisk}
  \label{fig:iddrvsxen-ramdisk-rdwr}
  \end{subfigure}\\
  \begin{subfigure}[b]{\textwidth}
  \includegraphics[width=\textwidth]{iddrvsxen-loop-rd}
  \caption{Random reads on a loop device}
  \label{fig:iddrvsxen-loop-rd}
  \end{subfigure}
\end{figure}

\begin{figure}[H]
  \ContinuedFloat
  \begin{subfigure}[b]{\textwidth}
  \includegraphics[width=\textwidth]{iddvsxen-harddisk-rd}
  \caption{Random reads on a SATA disk}
  \label{fig:iddrvsxen-harddisk-rd}
  \end{subfigure}\\
\caption{Interrupt-based IDDR system vs Xen split driver}\label{fig:seqloopdisk}
\end{figure}

\section{Interrupt-based IDDR System vs Spinning-based IDDR System}
We measure and compare the performance of the interrupt-based IDDR system
with the spinning-based IDDR system. To compare the performance of both
systems, we measure performance of the system by varying the number of
SysBench threads. The SysBench benchmark executes random reads and writes 
on a ramdisk, loop device, and a SATA disk.

\subsection*{Experimental Setup}
In both systems, the application domain is domain 0, and the driver domain
is domain U.  We create a ramdisk and insert the backend driver in the
driver domain.  We insert the frontend driver in the application domain.
We measure the performance of both systems on a loop device with the
same setup.

However, we were unable to set up PCI passthrough for the SATA disk,
preventing us from running the SATA controller driver in domain U.
For the SATA portion of the experiments, we use domain 0 as the
driver domain and domain U as the application domain, while still
retaining all other aspects of the IDDR implementation.

\subsubsection*{Random reads and writes}

\paragraph{Comparison:}

Figure~\ref{fig:rndramdisk}, Figure~\ref{fig:rndloopdisk}, and
Figure~\ref{fig:rndharddisk} compare the throughput of the interrupt-based
IDDR and the spinning-based IDDR system for read, write, and read/write
workloads using a ramdisk, loop device, and SATA disk, respectively.

Figure~\ref{subfig:rndrd-ramdisk}, Figure~\ref{subfig:rndrd-loopdisk},
and Figure~\ref{subfig:rndrd-harddisk} show that the spinning-based IDDR
system performs better when data is read from a device randomly.

Figure~\ref{subfig:rndwr-ramdisk}, Figure~\ref{subfig:rndwr-loopdisk}
and Figure~\ref{subfig:rndwr-harddisk} compare the performance of
the device when data is written randomly to a device. The graph shows
that initially the spinning-based IDDR system performs better than the
interrupt-based IDDR system, but as the number of SysBench threads increases,
the throughput of the spinning-based IDDR system decreases.

Figure~\ref{subfig:rndrw-ramdisk}, Figure~\ref{subfig:rndrw-loopdisk}
and Figure~\ref{subfig:rndrw-harddisk} compare the performance for mixed
random read and write workloads.

\begin{figure}[!ht]
 \centering
  \begin{subfigure}[b]{\textwidth}
  \includegraphics[width=\textwidth]{rndrd-ramdisk}
  \caption{Random reads}
  \label{subfig:rndrd-ramdisk}
  \end{subfigure}\\
  \begin{subfigure}[b]{\textwidth}
  \includegraphics[width=\textwidth]{rndwr-ramdisk}
  \caption{Random writes}
  \label{subfig:rndwr-ramdisk}
  \end{subfigure}
\end{figure}

\begin{figure}[H]
  \ContinuedFloat
  \begin{subfigure}[b]{\textwidth}
  \includegraphics[width=\textwidth]{rndrw-ramdisk}
  \caption{Random reads writes}
  \label{subfig:rndrw-ramdisk}
  \end{subfigure}
  \caption{Random reads and writes on a ramdisk}\label{fig:rndramdisk}
\end{figure}

\begin{figure}[!ht]
\centering
  \begin{subfigure}[b]{\textwidth}
  \includegraphics[width=\textwidth]{rndrd-loopdisk}
  \caption{Random reads}
  \label{subfig:rndrd-loopdisk}
  \end{subfigure}\\
  \begin{subfigure}[b]{\textwidth}
  \includegraphics[width=\textwidth]{rndwr-loopdisk}
  \caption{Random writes}
  \label{subfig:rndwr-loopdisk}
  \end{subfigure}
\end{figure}

\begin{figure}[H]
  \ContinuedFloat
  \begin{subfigure}[b]{\textwidth}
  \includegraphics[width=\textwidth]{rndrw-loopdisk}
  \caption{Random reads writes}
  \label{subfig:rndrw-loopdisk}
  \end{subfigure}
\caption{Random reads and writes on a loop device}\label{fig:rndloopdisk}
\end{figure}

\begin{figure}[!ht]
\centering
  \begin{subfigure}[b]{\textwidth}
  \includegraphics[width=\textwidth]{rndrd-harddisk}
  \caption{Random reads}
  \label{subfig:rndrd-harddisk}
  \end{subfigure}\\
  \begin{subfigure}[b]{\textwidth}
  \includegraphics[width=\textwidth]{rndwr-harddisk}
  \caption{Random writes}
  \label{subfig:rndwr-harddisk}
  \end{subfigure}
\end{figure}

\begin{figure}[H]
  \ContinuedFloat
  \begin{subfigure}[b]{\textwidth}
  \includegraphics[width=\textwidth]{rndrw-harddisk}
  \caption{Random reads writes}
  \label{subfig:rndrw-harddisk}
  \end{subfigure}
\caption{Random reads and writes on a SATA disk}\label{fig:rndharddisk}
\end{figure}

\paragraph{Observation:}
The performance analysis of the interrupt-based and spinning-based IDDR system
shows that initially the throughput of both systems increases and then it
remains constant. We measure the throughput of the system with varying
number of SysBench threads.

In both systems, when the number of SysBench threads is low,
the rate at which data is read and written is low and when the number of
SysBench threads is high, the rate at which data is read and written is
high.  Once maximum throughput is achieved, it remains constant.

In the interrupt-based IDDR system, virtual interrupts are sent between
application and driver domains whereas the spinning-based approach
attempts to avoid these virtual interrupts.  
Our optimization technique reduces the frequency of the
virtual interrupt being sent between domains. 
The higher performance gains at lower levels of concurrency
may be due to missed opportunities for batching in the interrupt-based case,
increasing the need for virtual interrupts even further.

\subsubsection*{CPU utilization}

The spinning-based IDDR system uses CPU cycles that could also be
used by applications unless excess CPU capacity is available (as in many
multicore or manycore systems). Hence we exploit a trade-off between higher 
CPU utilization and higher throughput.

In this section we compare the CPU utilization of the interrupt-based and
the spinning-based IDDR systems. Figure~\ref{fig:cpuramdisk}, Figure~\ref{fig:cpuloopdisk}, and
Figure~\ref{fig:cpuharddisk} compare the CPU utilization of both 
systems using the ramdisk, loop device and SATA disk setups.

\paragraph{Observation: }
In the spinning-based version of IDDR, the read response and request threads
spin continuously unless the driver is idle for extended periods of time.
Thus, up to 2 cores of the system may be consumed just for spinning.
Figures~\ref{fig:cpuramdisk} and~\ref{fig:cpuloopdisk} show the CPU
utilization for the ramdisk and loop device setups.
It can be seen that the CPU utilization of the spinning-based 
implementation ranges between 100\% and 250\%, which is within the
expected range.

\begin{figure}[!ht]
\centering
  \begin{subfigure}[b]{\textwidth}
  \includegraphics[width=\textwidth]{cpu-rndrd-ramdisk}
  \caption{Random reads}
  \label{subfig:cpu-rndrd-ramdisk}
  \end{subfigure}\\
  \begin{subfigure}[b]{\textwidth}
  \includegraphics[width=\textwidth]{cpu-rndrw-ramdisk}
  \caption{Random writes}
  \label{subfig:cpu-rndrw-ramdisk}
  \end{subfigure}
\end{figure}

\begin{figure}[H]
  \ContinuedFloat
  \begin{subfigure}[b]{\textwidth}
  \includegraphics[width=\textwidth]{cpu-rndwr-ramdisk}
  \caption{Random reads writes}
  \label{subfig:cpu-rndwr-ramdisk}
  \end{subfigure}
\caption{Comparison of CPU utilization - ramdisk}\label{fig:cpuramdisk}
\end{figure}

\begin{figure}[!ht]
  \begin{subfigure}[b]{\textwidth}
  \includegraphics[width=\textwidth]{cpu-rndrd-loopdisk}
  \caption{Random reads}
  \label{subfig:cpu-rndrd-loopdisk}
  \end{subfigure}\\
  \begin{subfigure}[b]{\textwidth}
  \includegraphics[width=\textwidth]{cpu-rndrw-loopdisk}
  \caption{Random writes}
  \label{subfig:cpu-rndrw-loopdisk}
\end{subfigure}
\end{figure}

\begin{figure}[H]
  \ContinuedFloat  \begin{subfigure}[b]{\textwidth}
  \includegraphics[width=\textwidth]{cpu-rndwr-loopdisk}
  \caption{Random reads writes}
  \label{subfig:cpu-rndwr-loopdisk}
  \end{subfigure}
\caption{Comparison of CPU utilization - loop device}\label{fig:cpuloopdisk}
\end{figure}

\begin{figure}[!ht]
\centering
  \begin{subfigure}[b]{\textwidth}
  \includegraphics[width=\textwidth]{cpu-rndrd-harddisk}
  \caption{Random reads}
  \label{subfig:cpu-rndrd-harddisk}
  \end{subfigure}\\
  \begin{subfigure}[b]{\textwidth}
  \includegraphics[width=\textwidth]{cpu-rndwr-harddisk}
  \caption{Random writes}
  \label{subfig:cpu-rndwr-harddisk}
  \end{subfigure}
\end{figure}

\begin{figure}[H]
  \ContinuedFloat
  \begin{subfigure}[b]{\textwidth}
  \includegraphics[width=\textwidth]{cpu-rndrw-harddisk}
  \caption{Random reads writes}
  \label{subfig:cpurndrw-harddisk}
  \end{subfigure}
\caption{Comparison of CPU utilization - SATA disk}\label{fig:cpuharddisk}
\end{figure}