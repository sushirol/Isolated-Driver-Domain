This chapter discusses work that is closely related to the IDDR system. 

Since our work focuses both on improving the reliability of a system through
device driver isolation and exploring opportunities to improve their performance,
we discuss work related to each of these aspects.
Section~\ref{sec:robustness} discusses work on improving reliability and
Section~\ref{sec:interdomain} discusses work which concentrates on
optimizing interdomain communication.

\section{Reliability}
\label{sec:robustness}

\subsection{Driver Protection Approaches}
Other researchers have recognized the need for device driver 
isolation~\cite{tanenbaum2006can, coveritykernel}. 
Common approaches include user-level device drivers, hardware-based 
driver isolation schemes, and language-based approaches.

Multiple implementations run device drivers in user
mode. Even though user mode device drivers allow user level
programming and provide good fault isolation between components,
they can suffer from poor performance~\cite{armand1991give} and also
require rewriting of existing device drivers~\cite{Leslie+:jcst2005}.

Microdrivers~\cite{Ganapathy:2008:DIM:1346281.1346303} extend the
user mode device driver approach by splitting a device driver into two
parts. Performance critical operations run inside the kernel and 
the rest of the driver code runs in a user mode process. 
Microdrivers can deliver better performance than pure user-level drivers.

Apart from user mode device drivers, some approaches use 
hardware-based mechanisms inside the kernel to isolate components. Nooks
is one example of such approaches~\cite{swift2005improving}.  
Nooks focuses on making device drivers less vulnerable to bugs and
malicious exploits within a traditional monolithic kernel.
It creates a lightweight protection domain around each device driver. 
A wrapper layer monitors
all interactions between the driver and the kernel, and protects the
kernel from faulty device drivers. Nooks requires device drivers to be
modified as their interaction with the kernel changes.

SUD~\cite{Boyd-Wickizer+:atc2010} runs unmodified Linux device drivers
in user space. It emulates IOMMU hardware to achieve this.
Running device drivers in user space allows the use of user-level
development and debugging tools.

Dune~\cite{Belay+:osdi12} is a system that provides an application with direct
and safe access to hardware features such as page tables, tagged
TLBs, and different levels of privileges. It uses virtualized hardware to isolate
applications from each other. Dune delivers hardware interrupts
directly to applications.
Dune isolates different applications from one another, but 
it does not directly isolate device drivers.

\subsubsection*{Virtualization Based Approaches}

Fraser et al~\cite{Fraser04safehardware} originally proposed
isolated driver domains for Xen.  IDDR adopts their idea, but provides
an entirely separate implementation, along with the optimizations described
in Section~\ref{}.

LeVasseur et. al.~\cite{LeVasseur04UnmodifiedDriverReuse} presents
a virtualization based system to reuse unmodified device drivers. 
In this approach, an unmodified device
driver is run with a kernel in a separate virtual machine, isolating
it from faults. The main goal of this system is to reuse device drivers 
across different operating systems. 

VirtuOS~\cite{Nikolaev:2013:VOS:2517349.2522719} is a 
solution that allows processes to directly communicate with kernel
components running in a separate domain at the system call level. 
VirtuOS exploits virtualization to isolate the components of
existing OS kernels in separate virtual machines. 
These virtual machines directly serve system calls from user processes.

\subsection{Kernel Designs}
Decomposing kernel functionality into separate components can provide
better fault containment. 
Microkernels such as Mach~\cite{Accetta+:usenix86} and L4~\cite{Liedtke+:sosp95} 
are examples of such an approach. 
Microkernel-based designs include
only essential functionalities such as memory management, interprocess
communication, scheduling, and low level device drivers inside the kernel.
All remaining system components, such as file system and process
management, are implemented inside user processes that communicate
via message passing.
% please cite L3/L4 Heiser SOSP 2013

Microkernels and hypervisors provide similar abstractions\cite{are microkernels ... done right 2papers}.
A Microvisor~\cite{Heiser+:acm10} is a microkernel design that allows
the execution of multiple virtual machines.

\section{Interdomain Communication}
\label{sec:interdomain}
Related work has also focused on improving interdomain communication.

In Xen VMM, a domain communicates with the privileged domain
through the split device driver mechanism~\cite{Fraser04safehardware},
which incurs the copying and page flipping overheads discussed in Section~\ref{}.
In order to overcome the page flipping performance overhead, 
Zhang et al~\cite{Zhang:2007:XHI:1516124.1516138} proposed a UNIX domain socket 
like interface called XenSocket, which provides high throughput 
interdomain communication.  The use of XenSockets requires a specific API,
so frontend and backend drivers cannot benefit from it transparently.

Fido~\cite{Burtsev:2009:FFI:1855807.1855832} is a shared memory
based interdomain communication mechanism. Fido speeds up
interdomain communication by reducing data copies in the
Xen hypervisor. In contrast, the IDDR system improves the interdomain
communication mechanism of split device drivers by avoiding context
switches. Fido achieves its goals of improving communication performance
by sacrificing some security and protection guarantees.

