In this thesis we presented the Isolated Device Driver (IDDR) system. The IDDR system is an operating system which provides isolation between a device driver and the Linkux kernel components by running device driver in the driver domain. The IDDR system is a re-implementation of the Xen's isolated driver domain. 
\\[3mm]
In Xen's isolated driver domain, a domain communicates with the device driver running in the priviledged domain through a split device driver mechanism. The split device driver uses interrupt based model for notification purpose. We replace the interrupt based model with the spinlock based model. This avoids the unwanted domain rescheduling for every software interrupt. The IDDR system trades cpu cycles for the performance benefit.
\\[3mm]
The IDDR system will be beneficial when used with a data intensive applications. As in the data intesive applications, when the workload is low, the CPU utilization is low. Hence the IDDR system can afford to waste the idle CPU cycles for performance benefit. On the other hand, in case of heavy workload, the CPU utilization is high. Hence if the IDDR system utilizes the CPU cycles in order to improve the performance, it is still acceptable as those CPU cycles would have anyways used. However, the IDDR system will worsen the performance of the system, if the system is running CPU intensive applications with an average workload.
% \\[3mm]
% In current implementation of the IDDR system, we copy data from an user application buffer to shared memory, and then from shared buffer to the device driver. It will be interesting to see the performance improvement if the data copy is eliminated.  